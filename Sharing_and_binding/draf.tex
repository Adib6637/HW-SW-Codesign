draf

abstarct

Introduction

graph
	sequencing graph. 

resources 

constraint
	
		implementation constraints reflect the desire of the designer to achieve a structure with some properties. Examples are 	area constraints and performance constraints, e.g., cycle-time andlor latency bounds. A different kind of implementation constraint is a resource b~nding constraint. In this case, a particular operation is required to be implemented by a given resource. 
These constraints are motivated by the designer's previous knowledge, or intuition, that one particular choice is the best and that other choices do not need investigation. Architectural synthesis with resource binding constraints is often referred to as synthesis from partial structure


problem
	
	The Temporal Domain: Scheduling
	The Spatial Domain: Binding 
	Hierarchical Models 

Approach


Resource dominated circuit

	non hierarchical sequence graph
		compatibility graph
			equation
			model
		conflict graph
			complement of compatibility graph
			model(vertex color)
			clique
		conflict graph as an interval
			execution interval
			intersection between two interval
			minimum vertex coloring in polynomial time
		
	resource sharing and binding in sequence graph
		model call
			single call
			multiple call
		iteration
			unroll
			similar to model call
		branching
		
		
		
General circuit

	register sharing
		lifetime variable
		variable alive in non overlaping intervals or under alternative condition are compatible
		compatibility graph
		conflict graph
		non-hierarchical
		hierarchical
	multiple memory binding
	
	bus sharing and bnding
	
	multiplexer
	
	